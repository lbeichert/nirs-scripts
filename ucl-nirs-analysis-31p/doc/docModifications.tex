\documentclass[12pt]{article}

\usepackage{hyperref}
\usepackage{dirtree}
\parindent 0pt

\title{Modifications to ucl-nir-analysis in order to include 31P-data}
\author{L.Beichert}
\date{July 2014}

\begin{document}
\maketitle

The modified version of the synchronisation script requires the same folder structure as the original ucl-nirs-analysis script with an additional subfolder \url{/phosphorus} to every \url{LWPXXX} folder containing the 31P data sheets. 

\dirtree{%
.1 pigletdatadir.
.2 studyDir.
.3 LWPXXX.
.4 nirs.
.4 systemic.
.4 phosphorus.
.5 LWPXXX\_insult\_analysis$.$xlsx.
.3 LWPXXY.
.3 LWPXXZ.
.3 study\_log$.$xlsx.
.3 31p\_log$.$xslx.
.3 $\dots$.
}


\paragraph{}
In order to read out the 31p data and synchronise it with the nirs/systemic data, the following modifications must be made to the folder containing the original ucl-nirs-analysis scripts:

\begin{itemize}
\item Copy the new file \emph{phosAnalyseAndSync.m} to the original \url{/func} folder.
\item Also in the original \url{/func} folder, replace the original \emph{load31PData.m} with the new version
\item Use the included files \emph{DataAnalyse\_31p.m} and \emph{31p\_log.xlsx} as templates and adjust them to the files you want to analyse and the folder topology on your PC.
\item (If script runs very slowly: comment out lines 345-349 in \url{/tarragona/Broadband_Tarragona.m} )
\end{itemize}

\paragraph*{}
Running \emph{DataAnalyse} will then create a variable \emph{pigDataPTB} that has the same basic structure as \emph{pigDataNTB} with additional rows of data for various MRS variables. 
This variable is stored in a file called \emph{pigDataPTB} in the specified output folder.

Note that for the synchronisation NIRS and systemic data are down-sampled to the sampling time of the 31p-data (usually 1min).

\end{document}