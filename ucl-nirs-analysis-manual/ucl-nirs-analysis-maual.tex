\documentclass[10pt]{article}

\usepackage{hyperref}
\usepackage{dirtree}
\usepackage{listings}
\lstset{basicstyle=\scriptsize}
\usepackage{rotating}
\parindent 0pt

\title{Synchronisation of NIRS and Systemic Data}
\author{L.Beichert}
\date{July 2014}

\begin{document}
\maketitle

Quick manual of how to get the script running that converts .tdf and systemic files into one combined matlab array.

\paragraph*{Script Files}
Download scripts from \url{http://www.github.com/aot/ucl-nirs-analysis}. 
Unzip into MATLAB directory where you keep your scripts.

%\paragraph{}
\dirtree{%
.1 MATLAB.
.2 ucl-nirs-scripts.
.3 func.
.3 tarragona.
.3 DataAnalyse$.$m.
}


\paragraph{Data Files}
For the data files the following folder structure is expected:

%\paragraph{}
\dirtree{%
.1 pigletdatadir. 
.2 studydir.
.3 LWPxxx.
.4 nirs.
.4 systemic.
.4 phosphorus.
.3 LWPxxy.
.3 \dots.
.3 LWPxxz.
.3 output.
.3 stud\_log(excelSheet).
.3 31p\_log (excelSheet).
.3 \dots .
}

where each LWPxxx folder contains the nirs, systemic and phosphorus files in its according subfolders. 'stud\_log.xlsx' is the study log taken during experiments. 
It is not directly accessed by the scripts but is needed to create the sheets that are.

\paragraph{Appdata and Path variables}
For the scripts to run the appdata varible 'pigletdatadir' must be set to the according path by creating a file \emph{'startup.m'} in the 
\href{http://www.mathworks.co.uk/help/matlab/matlab_env/matlab-startup-folder.html}{MATLAB startup folder} that contains: 

\begin{lstlisting}
% adjust path according to your own PC
setappdata(0,'pigletdatadir', 'C:\User\pigletdatadir\');
\end{lstlisting}

Also the subdirectories 'func' and 'tarragona' of the ucl-nirs-analysis folder must be added to the MATLAB path. This can be done by adding to the \emph{'startup.m'} file:

\begin{lstlisting}
% adjust paths according to your own PC
addpath('C:\MATLAB\ucl-nirs-analysis\func');
addpath('C:\MATLAB\ucl-nirs-analysis\tarragona');
\end{lstlisting}

\paragraph*{Creation of Specific Log File}
In order to synchronise the various data sources the script requires a spreadsheet containing the names of the files that are synchronised and the start time of the measurement (see example files for exact format). 
This file is excpected in the studydir, it makes sense to name it like '31p\_log.xlsx' depending on the kind of data that is supposed to be analysed.

\paragraph{Modifying the DataAnalyse Script}
Then, the best way to proceed is to duplicate the example 'dataAnalyse.m' file and rename the new file to, for example, '31p\_DataAnalyse.m'. 
This new script file must then be modified:
\begin{itemize}
\item ln.4: studyDir=... replace 'post-conditioning' with the name of the current studydir
\item ln.21: fileSuffix=... change file suffix according to data analysed, for example to '31p'
\item ln.23: excelExpLogFile=... replace 'eeg\_log.xlsx' with name of according log file, for example '31p\_log.xlsx'
\item you may have to comment out line 11 (csch=...)
\end{itemize}

\paragraph{Output}
Finally, run the script. Once it is done, two variables are created that contain the data. \emph{pigData} contains NIRS data only and \emph{pigDataNTB} contains synchronised NIRS and systemic data. 

It makes sense to directly save these variables to a file by inserting the following line at the end of the new 'dataAnalyse' file:
\begin{lstlisting}
% saves the variables pigData and pigDataNTB to a file called pigData.m
save([outputDir,filesep,fileSuffix,filesep,'pigData.m'], 'pigData', pigDataNTB');
\end{lstlisting}
\end{document}
